%==================================================================
%
%  LaTeX kommandoer for FFI malen (ffiarticle)
%
%==================================================================

%===================================================================
%
% Forside
%
%===================================================================

% Forfatter 
% Settes inn med standard \author{} i hoveddokument
 
% Tittel
% Settes inn med standard \title{} i hoveddokument

% Undertittel 
%\subtitle{ (undertittel) } % Norsk dersom norsk spr�k er valgt, ellers engelsk 

% Engelsk tittel
\englishtitle{(englishtitle)} % Sett inn tekst KUN dersom norsk spr�k er valgt
% NB! dersom engelsk spr�k settes tittel inn med \title{} i hoveddokument 


%====================================================================
%
% Gradering av tittel 
%
%====================================================================

%\unclassifiedtitle % Brukes for � gj�re tittelen ugradert i et gradert dokument
%\restrictedtitle % Brukes for � gj�re tittelen begrenset i et gradert dokument

%====================================================================
%
% Paragrafer i offentlighetsloven. Brukes for dokumenter gradert til
% unntattOffentlighet eller fortrolig 
%
%====================================================================

%\offentlighetsloven{\S5a, jfr forvaltningsloven \S13}

%\offentlighetsloven{\S\ 13 jf fvl \S\ 13}% (Taushetsplikt)}                         
%\offentlighetsloven{\S\ 14}% (Organinterne dokumenter)}                          
%\offentlighetsloven{\S\ 15.1}% (Interne dokumenter utanfr� - underordna organ)}  
%\offentlighetsloven{\S\ 15.2}% (Interne dokumenter utanfr� - r�d og
%vurderingar)}
%\offentlighetsloven{\S\ 20}% (Utanrikspolitiske interesser)}                     
%\offentlighetsloven{\S\ 21}% (Nasjonale forsvars- og tryggingsinteresser)}       
%\offentlighetsloven{\S\ 22 - f�rste punktum}% (Budsjettsaker - departementet)}   
%\offentlighetsloven{\S\ 22 - annet punktum}% Budsjettsaker-                      
                                % underliggende organ                           
%\offentlighetsloven{\S\ 23.1}%�konomi, l�nns- og personalforvaltnign             
%\offentlighetsloven{\S\ 23.3}%Anbud                                              
%\offentlighetsloven{\S\ 25}%Tilsetjingssaker med mer   

%=========================================================================
%
% Nedgraderingstid for et gradert dokument
%
%=========================================================================

%\downgrad{5 YEARS}


%=========================================================================
% 
% Datoer
%
%==========================================================================

\dateofpublishingNo{Dag M�ned �r}
\dateofpublishingEn{\today}

%==========================================================================
%
% Tittelsidas bakside
%
%==========================================================================

% Rapportnummeret
\reportnumber{��rr/yyyyy}

% Prosjektnummer
\projectnumber{xxx}

% ISBN-nummer
%\isbnP{ISBN papirversjon}
%\isbnE{ISBN elektronisk versjon}

% Emneord

\emneord{Neutral gas}
\emneord{Large eddy simulation}
\emneord{Urban street cayon}
\emneord{Fluent performance test}
\emneord{Fluent vs. CDP}
\emneord{Fluent solver settings}

%============================================================================
%
% Sjefer
%
%============================================================================

%Adm. dir.
\directorgeneral{John Mikal St�rdal}
%Avdelingssjef
\divisionmanager{(avdelingssjef)}
%Forskningssjef
%\chiefscientist{(forskningssjef)}
%Prosjektleder
\projectleader{(prosjektleder)}


%============================================================================
% 
% Sammendrag
%
%============================================================================

% Norsk sammendrag

\sammendrag{
Denne rapporten unders�ker hvor godt den kommersielle CFD-programvaren Fluent kan estimere dispersjon 
av en n�ytral gass. Resultatene er sammenlignet med data fra et vindtunnel-eksperiment
og tilsvarende simuleringer i CDP\@. Alle simuleringer er utf�rt ved hjelp av Large Eddy Simulations
(LES). Det er ogs� eksperimentert med ulike diskretiseringsskjemaer som ligger tilgjengelig i
Fluent, effekten av en subgrid-scale (SGS) modell og bekreftelse av at n�yaktigheten �ker med antall 
frihetsgrader. Testproblemet som er brukt for sammenligning av data
er dispersjon av en n�ytral gass i en vindtunnel med fire kubiske klosser som representerer et urbant
bygningsmilj�.

I byplanlegging og st�rre prosjekter innenfor infrastruktur er det viktig � ha kunnskap til hvordan 
farlige gasser og v�sker b�r behandles. � kunne simulere disse scenarioene tillater oss � v�re 
bedre forberedt p� u�nskede hendelser som gasslekkasje fra en fabrikk. Valg av programvare og 
matematisk formulering er viktig for � kunne tilby en korrekt l�sning innenfor en akseptabel 
tidsramme.

Resultatene fra Fluent stemmer godt overens med de eksperimentelle dataene og simuleringene gjort
i CDP\@. Bredden og h�yden p� gass-skyen som oppst�r som en f�lge av gassutslippet 
er konsistent med tidligere data. Ulike diskretiseringsskjemaer har et signifikant utslag p�
regnetiden og resultatene varierer ogs� i noen grad. 
Selv om tid kan bli spart ved � anvende andre innstillinger enn de som er anbefalt av Fluent, 
tilsier resultatene at standardinstillingene i Fluent gir de mest n�yaktige resultatene.
Simuleringene har v�rt gjennomf�rt p� opptil 120 kjerner og koden skalerer godt
etterhvert som antall kjerner �ker.

}

% English summary
\abstractdocpage{This report investigates how well the commercial CFD software 
Fluent can estimate dispersion of a neutrally buoyant gas. 
The results obtained are compared with data from a wind-tunnel 
experiment and similar simulations performed in CDP\@. All simulations are performed using 
Large Eddy Simulation (LES). It is also experimented with different discretization schemes 
available in Fluent, the effect of a subgrid-scale model as well as confirming that the
accuracy increases with the degrees of freedom.
The reference case used to compare the data is gas dispersion 
of a neutral gas in a wind-tunnel with four cubic obstacles representing an urban street canyon. 

In city planning and large projects within infrastructure it is important to have knowledge of how 
hazardous gases and liquids should be handled. Being able to simulate these scenarios enable us 
to be better prepared for unwanted events such as a factory gas leak. The choice of 
software and mathematical formulation is important in order to provide a correct solution 
within a reasonable amount of time.

The results obtained using Fluent agree well with the experimental data and the CDP simulations, 
The width and height of the plume generated by the gas release is consistent with the previous 
results. There are some disagreements in the data measured close to the wall.
The choice of discretization scheme has a significant impact on the computational time 
and the results vary to some degree. Allthough the time can be saved by applying a non-default
scheme the results indicate that the default setting in Fluent provides the most accurate solution.
The simulations have been performed on up to 120 cores and the program scales very well as the amount of cores is increased.
}
% Forord
%\forord{Forord er valgfritt}
