% Appendix Template

\chapter{Variables and Functions in Nek5000} % Main appendix title

\label{AppendixB} % Change X to a consecutive letter; for referencing this appendix elsewhere, use \ref{AppendixX}

\lhead{Appendix B. \emph{Variables and functions in Nek}} % Change X to a consecutive letter; this is for the header on each page - perhaps a shortened title

\section{Variables}

The Nek manual provides information on many of the variables given in the .rea and SIZE
file. It is however no list of useful variables defined in other files. 
In~\tref{tab:variables} some of the variables that have been frequently used in .usr
subroutines are listed. Note that all the variables listed in this table are defined outside of SIZE, .rea and .usr 


\begin{table}
    \centering
    \begin{tabular}{l l}
        \multicolumn{2}{l}{\textbf{AVG}}\\ \hline
uavg(ax1,ay1,az1,lelt)           & Averaged values of u, similar for v,w,p\\
urms(ax1,ay1,az1,lelt)           & Variance of u, similar for v,w,p\\
vwms(ax1,ay1,az1,lelt)           & Covariance of vw\\
tavg(ax1,ay1,az1,lelt,ldimt)     & Averaged values of t and all passive scalars\\

\multicolumn{2}{l}{\textbf{GEOM}}\\ \hline
xm1(lx1,ly1,lz1,lelt)            & X-coordinates for the velocity mesh \\
xm2(lx2,ly2,lz2,lelv)            & X-coordinates for the pressure mesh\\
unx(lx1,lz1,6,lelt)              & Surface normals \\
area(lx1,lz1,6,lelt)             & Surface integral weights\\

\multicolumn{2}{l}{\textbf{INPUT}}\\ \hline

cbc(6,lelt,0:ldimt1)             & Boundary conditions of each face\\
ccurve(12,lelt)                  & Curved side character \\
curve(12,6,lelt)                 & Curved side information\\

\multicolumn{2}{l}{\textbf{PARALLEL}}\\ \hline
lglel(lelt)                      & 	Mapping from local to global element index\\
gllel(lelg)                      & 	Mapping from global to local element index\\

\multicolumn{2}{l}{\textbf{SOLN}}\\ \hline
vx(lx1,ly1,lz1,lelv)            & X-velocity\\
t(lx1,ly1,lz1,lelv,ldimt)       & Temperature and passive scalars\\
vtrans(lx1,ly1,lz1,lelt,ldimt1) & Diffusive constant to additional scalars\\
vdiff(lx1,ly1,lz1,lelt,ldimt1)  & Convective constants to additional scalars\\

\multicolumn{2}{l}{\textbf{TSTEP}}\\ \hline
istep                            & Current iteration step\\
iostep                           & Output step frequency\\
time                             & Current time\\
tstep                            & Current time step	\\
dt                               & Time step\\
dtlag(10)                        & The previous 10 time steps\\
bd(10)                           & Max 10 backward difference coeffs\\
ab(10)                           & Max 10 extrapolation coeffs (Adam-Bashforth)\\

\multicolumn{2}{l}{\textbf{WZ}}\\ \hline
zgm1(lx1,3)                      & GLL points for x,y and z directions \\

\multicolumn{2}{l}{\textbf{OTHER}}\\ \hline            
x,y,z                            & Local coordinates assigned by nekasgn()\\
ux,uy,uz                         & Local velocities assigned by nekasgn()\\
temp                             & Local temperature assigned by nekasgn()\\
nio                              & Processor node number  \\
ndim                             & Number of dimensions\\
nelv                             & Number of elements for velocity mesh\\
nelt                             & Number of elements for the t-mesh\\
pm1 (lx1,ly1,lz1,lelv)           & Pressure mapped to mesh 1\\

\end{tabular}

\caption{useful variables in Nek, the bold capital sections denote the separate files in /trunk/nek/.}

\label{tab:variables}
\end{table}

\section{Functions}

\subsection{standard calculations found in math.f or navier1.f}

\textbf{nekasgn(ix,iy,iz,ie) } Assigns the coordinates of node 
    (ix,iy,iz) in element ie to the common variables x,y,z 

\textbf{facind(kx1,kx2,ky1,ky2,kz1,kz2,nx1,ny1,nz1,f) } 
Assigns the index limits of a face f with nx1,ny1,nz1 points in each spatial direction.

\textbf{zwgll(zg,wg,nx1)} Get the nx1 GLL-points and weights to zg and wg.

\textbf{cadd(zg,c,nx1)} Adding a constant c to a vector zg of length nx1.

\textbf{cmult(zg,c,nx1)} Multiplying every element of vector zg of length nx1 with c.

\textbf{chsign(wrk,nx1)} change the sign of every element in vector wrk of length nx1.

\textbf{cfill(zg,c,nx1)}  Fill vector zg of length nx1 with the constant c.

\textbf{rzero(zg,nx1) } Fill vector zg of length nx1 with zeroes.

\textbf{rcopy(zg,zg2,nx1) } copy all elements from vector zg2 to vector zg, both of length nx1.

\subsection{Functions regarding mesh and distribution of GLL-points}

\textbf{gh\_face\_extend(x,zg,n,type,e,v)}  The Gordon hall algorithm described in chapter~\ref{numerics}, the type variable denotes whether the 
algorithm should use vertices, edges or faces to distribute the inner GLL-points. 

\textbf{xyzlin(xl,yl,zl,nxl,nxl,nxl,e,ifaxl)} Generate bi- or trilinear mesh.

\textbf{fix\_geom()} Routine for assuring mesh consistency and re distributing the internal  gll-points correctly on the updated geometry.

\subsection{Additional auxiliary functions implemented for this thesis}

\textbf{fix\_gll(e,f)} Redistribute the gll-points between the given face and the opposite to make sure that all points lie within the element.

\textbf{getface(kx1,kx2,ky1,ky2,kz1,kz2,wrk,n,e)} assigning the values of the face in element e corresponding to the index limits kx1,kx2\ldots to the array wrk of size n*n*3.

\textbf{getsurfnorm(sn,ix,iy,iz,f,ie)} Providing the surface normal sn at point ix,iy,iz of element ie and face f 

\textbf{calcerror(error,lambda,sn,wrk,radius)} calculate the distance from the initial gll-point to a given point on the surface.  

\textbf{interp\_up(iinterp,rinterp,error,k)} Update interpolation points

\textbf{set\_new\_pt(iinterp,rinterp,ix,iy,iz,e)} defining the position of the new gll-point on the surface

\textbf{getlimits(k,n,kx1,kx2,ky1,ky2,kz1,kz2)} Get the index limits kx1,kx2\ldots corresponding to edge k with n gll-points.

\textbf{setcoords(xq,yq,zq,xedge,yedge,zedge,nxl,k)} copy the updated edges xedge to the initial edges xq

\textbf{getcoords(xq,yq,zq,xedge,yedge,zedge,nxl,k)} copy the node information from the initial edge xq to xedge
