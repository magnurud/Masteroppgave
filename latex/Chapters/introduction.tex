% Chapter Template

\chapter{Introduction} % Main chapter title

\label{introduction} % Change X to a consecutive number; for referencing this chapter elsewhere, use \ref{ChapterX}

\lhead{Chapter 1. \emph{Introduction}} % Change X to a consecutive number; this is for the header on each page - perhaps a shortened title

%----------------------------------------------------------------------------------------
%	SECTION 1
%----------------------------------------------------------------------------------------

%The Navier-Stokes equations have been studied for centuries and describes any fluid in motion. Depending on the properties of the fluid 
%and the circumstances of the flow different variants of these equations are studied. This thesis will restrict itself to the numerical 
%solutions of the incompressible N-S equations.  

The physics regarding fluids in motion are described mathematically by the Navier-Stokes (N-S) equations. 
They are a result of the conservation of momentum and mass and are stated in~\eref{eq:NS}.
This thesis is restricted to numerical solutions of the incompressible N-S equations. For a complete 
description of the necessary assumptions and simplifications it is referred to~\cite{White}.  
An important dimensionless quantity is the Reynolds number $Re$, which describes the ratio between 
momentum forces and viscous forces. For large Reynolds numbers the flow becomes turbulent and a 
large range of scales needs to be resolved. A lot of research has been devoted to determine the
amount of energy present at the different scales of motion, and the interaction between them. 
These ideas has led to turbulence modelling which is based on the 
idea that the effect of the smallest turbulent motions can be modelled, while 
the larger motions are resolved by the numerical grid. 
In this thesis both laminar and turbulent flows will be solved numerically, 
and a physically motivated model for the smallest structures will be compared
with a mathematical filter meant to stabilize the flow.
In addition to solving the N-S equations the 
transport of a passive scalar will also be analysed and compared with a set of reference solutions. 
The software applied in this thesis is Nek5000 which is an implementation of the spectral element method initialized in the 80's.
In addition to validate Nek5000 as a software for analyzing gas dispersion the work in this thesis 
also attempts to make Nek5000 more applicable to cases consisting of more complex geometry. 

This thesis is divided in 3 parts. The first part which consists of the two following chapters are devoted to the physical understanding 
of \eref{eq:NS}, the solution methods applied and a thorough presentation of the Spectral Element Method. \cref{nek} gives the reader a brief 
introduction to the functionalities of the solver Nek5000 to motivate some of the implementation performed. The last three chapters 
describes the work performed by the author, a presentation of the results and the discussion of these.

\section{A brief overview of the work done for this thesis}

Before we end this introduction a brief overview of the work done for this thesis is listed here. 
It can be divided in three main sections and will be presented properly in~\cref{implementation}
and~\ref{results}. 

\begin{itemize}
    \item Nek5000 grid generation
        \begin{itemize}
            \item Project edges onto a higher order arc.
            \item General surface projection.
            \item Testing of algorithm for different surfaces.
            \item Curvature propagation.
            \item Changes and adjustments to \verb|mshconvert|.
        \end{itemize}
    \item Laminar flow around cylinder
        \begin{itemize}
            \item Generation of geometry and grid.
            \item Setup of the problem with necessary input and output.
            \item Simulations with Nek5000 with different grid size, polynomial order, time stepping method and other settings.
        \end{itemize}
    \item Turbulent flow with gas dispersion
        \begin{itemize}
            \item Grid generation.
            \item Setup the problem with necessary input and output.
            \item Implementation of the spatial averaging routine for the dynamic Smagorinsky model.
            \item Simulations with Nek5000 with different grids, polynomial order,stabilization methods.
        \end{itemize}
    \item Analytical work with the numerical filter.
\end{itemize}

