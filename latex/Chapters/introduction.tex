% Chapter Template

\chapter{Introduction} % Main chapter title

\label{introduction} % Change X to a consecutive number; for referencing this chapter elsewhere, use \ref{ChapterX}

\lhead{Chapter 1. \emph{Introduction}} % Change X to a consecutive number; this is for the header on each page - perhaps a shortened title

%----------------------------------------------------------------------------------------
%	SECTION 1
%----------------------------------------------------------------------------------------

%The Navier-Stokes equations have been studied for centuries and describes any fluid in motion. Depending on the properties of the fluid 
%and the circumstances of the flow different variants of these equations are studied. This thesis will restrict itself to the numerical 
%solutions of the incompressible N-S equations.  

The physics regarding fluids in motion are described mathematically by the Navier-Stokes (N-S) equations. 
They are a result of the conservation of momentum and mass, and it is referred to~\cite{White} for a complete 
description of the necessary assumptions and simplifications. 
This thesis is restricted to numerical solutions of the incompressible N-S equations.
Without further introduction the incompressible N-S equations are stated as  
%
\begin{align}
    \begin{split}
    \frac{\partial \mathbf{u}}{\partial t} + \mathbf{u}\cdot \nabla\mathbf{u} &= 
    -\nabla p + \frac{1}{Re} \Delta\mathbf{u} + \mathbf{f}, \\
		\nabla \cdot \mathbf{u} &= 0.
    \end{split}
	\label{eq:NS}
\end{align}
%
$\mathbf{u}$ and $p$ denotes the velocity of the fluid and the pressure, while $\mathbf{f}$ is some external force. 
The Reynolds number $Re$ describes the relation between the viscous scales and the mean stream velocity scale.
For large Reynolds numbers the flow becomes turbulent and a large range of scales needs to be resolved. A lot 
of research has been devoted to determine the amount of energy present at the different scales of motion, and 
the interaction between them. This approach to \eref{eq:NS} leads to turbulence modelling which is based on the 
idea that the effect of the smallest turbulent motions can be modelled while the larger motions are resolved by 
the numerical grid. 
In this thesis both laminar and turbulent flows will be solved numerically, and a turbulence model will also 
be compared with a mathematical filter meant to stabilize the flow. In addition to solving the N-S equations the 
transport of a passive scalar will also be analysed and compared with a set of reference solutions. 
The software applied in this thesis is Nek5000 which is an implementation of the spectral element method initialized in the 80's.
In addition to validate Nek5000 as a software for analyzing gas dispersion it is also as a part of this thesis 
attempted to make Nek5000 more applicable to cases consisting of more complex geometry. 

The thesis is divided in 3 parts. The first part which includes the two following chapters are devoted to the physical understanding 
of \eref{eq:NS}, the solution methods and a thorough presentation of the spectral element method. \cref{nek} gives the reader a brief 
introduction to the functionalities of the solver Nek5000 to motivate some of the implementation performed. The last two chapters 
describes the work performed by the author, first a presentation of the different cases followed by the results and the discussion
of these.

