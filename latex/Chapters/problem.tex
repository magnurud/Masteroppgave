% Chapter 3 - Description of problem

\chapter{Application of nek5000 and the creation of grid} % Main chapter title

\label{problem} % Change X to a consecutive number; for referencing this chapter elsewhere, use \ref{ChapterX}

\lhead{Chapter 3. \emph{application of nek}} % Change X to a consecutive number; this is for the header on each page - perhaps a shortened title

%----------------------------------------------------------------------------------------
%	SECTION 1
%----------------------------------------------------------------------------------------
\section{Nek5000}

There are many numerical solvers for turbulent flows available on the market. From large commercial softwares such as fluent which runs as a 
black-box solver, to CDP which is a bit more specialized solver with the posibility to modify the code to apply it to the wanted usage and 
to full open-source codes such as nek5000 and openFOAM. The solvers can vary both in the fundamental numerical method (FV,FD,FEM,SEM) 
the implemented algorithm (Fractional Step, Poisson pressure, Uzawa) and the type of simulation (RANS,LES,DNS). 
The software is chosen depending on the users requirement, but the work flow does follow a certain structure in all cases. 

\begin{itemize}
	\item Practical problem --> Mathematical formulation --> Generation of mesh --> Simulation --> post-processing
\end{itemize}
An example is waterflow in a channel, which would be the practical problem. The mathematical formulation would be the incompressible N-S eq, 
with LES dynamical SGS. The mesh could consist of tetrahedros or hexahedrons. The simulation is the performed on some cluster before the 
wanted data in some sub-domain is visualized and presented. 

Nek provides a basic tool for generation of mesh. For more complex geometries this tool cannot compare with more visualized-based softwares 
such as ICEM from ANSYS which exports mesh to several numerical solvers such as Fluent and nastran.
It is therefore very useful to have an automatic way of converting a mesh created in ICEM to the format required by nek5000. 
In order for nek to run optimally the elements should be as homogenous and as similar to the reference element as possible. 
It is therefore of great interest to be able to propagate curved geometries into the neighbour-elements in order to have a smooth as 
possible transition from a boundary with high curvature.

\section{Gas dispersion in a simplified urban area}
The problem investigated in this work is gas dispersion of neutral gas in a velocity field through four cubic blocks.
Similar simulations have been done in CDP and Fluent which are compared to data from a wind-tunnel experiment performed by ALAN.
\begin{itemize}
	\item Description of problem and domain
	\item mesh and fluent/CDP mesh
\end{itemize}

\section{Creation of the mesh-convertion script}

\begin{itemize}
	\item initial script
	\item changes and modifications
	\item performance testing
	\item pitfalls
\end{itemize}
