% Chapter 2 - physics theory

\chapter{Fluid dynamics and problem specific mathematics} % Main chapter title

\label{physics} % Change X to a consecutive number; for referencing this chapter elsewhere, use \ref{ChapterX}

\lhead{Chapter 2. \emph{background on fluid dynamics}} % Change X to a consecutive number; this is for the header on each page - perhaps a shortened title

%----------------------------------------------------------------------------------------
%	SECTION 1
%----------------------------------------------------------------------------------------
\section{Navier-Stokes equation}

The physics regarding fluids in motion are described mathematically by the Navier-Stokes equation. The equations can be derived in many ways, 
and it is referred to \cite{White} for a complete explication. The general idea is to conserve momentum and mass in a control domain and by 
applying Reynolds transport theorem and the result is the following two equations reffered to as the N-S equations. 

\begin{align}
	\frac{\partial \mathbf{u}}{\partial t} + \mathbf{u} (\nabla \cdot \mathbf{u}) &= -\nabla p + \nu \nabla \cdot \tau \\
		\nabla \cdot (\rho \mathbf{u}) &= 0
	\label{eq:NS}
\end{align}

These equations have been studied for centuries and different physical situations lead to different simplifications and sets of equations. 
Examples are the Euler equations, Stokes problem, darcy flow and the incompressible navier-stokes equations. 
Without doing the derivation of the equations a short discussion of each term in equation N-S will be given here in order to understand there
physical origin and their mathematical attributes. 

The term $\partial \mathbf{u} /\partial t$  is the time-derivative of the flow, for a steady state flow this term will be equal zero.
The discretization of this term is often based on some implicit scheme.  

The convective term $\mathbf{u} (\nabla \cdot \mathbf{u})$ describes the transport due to the flow itself on each of its components. 
This term is non-linear and does therefore require the equations to be solved by some iteration procedure such as Newton iterations.

The diffusive term $\nu \Delta \mathbf{u}$ describes the natural diffusion of the fluid. The effect of diffusion is determined by the 
viscosity of the fluid. Mathematically this term makes the equations numerically stable and it is therefore generally easier
to solve the N-S equations for high-viscosity fluids. 

The pressure gradient $\nabla p$ is the only term involving the pressure, \ldots DISCUST THE MEANING OF PRESSURE\ldots 
in modern solution techniques it is common to solve eq. \ref{eq:NS} 
in a decoupled manner. 

The reynolds stress tensor $\tau = \cdots$ is the turbulent term of the N-S equation. In laminar flow this term can 
be neglected.

The second equation in \ref{eq:NS} imposes a divergence free constraint on the solution $\mathbf{u}$ if the fluid is 
assumed to be incombressible. The density $\rho$ will be a constant and the simplification follows trivially.

The coefficient $\nu$ is the inverse Reynolds number $1/Re$. The reynolds number can be understood in many different ways, but perhaps the most 
import role is that it describes the relation between the biggest length scales of the flow and the viscous length scales. Turbulent flows are
characterized by high reynolds number. The huge span in length scales requires an extremely fine mesh if the equations \ref{eq:NS} 
are to be solved exactly. Because a fine mesh implies a high computational cost a direct numerical solution (DNS) is not feasible for 
problems of a certain complexity.

\begin{itemize}
\item The time-derivative
\item The convective term 
\item The diffusive term 
\item The pressure gradient
\item The reynolds stress tensor
\item The reynolds number
\item The mass equation
\end{itemize}


\section{Resolving the turbulent term}
Depending on the wanted accuracy of your solution and the number of computational units at hand the turbulent term can be solved in 
different ways. A brief overview of the different approaches to the turbulent term are discussed in the following subsections

\begin{itemize}
	\item Laminar flow
	\item RANS
	\item LES
	\item K-epsilon and K-omega
	\item DNS
\end{itemize}

\section{Solution methods of incombressible N-S}
A non-linear set of equations requires a non-trivial solution method, and when the domain of the problem can be anyting from a simple channel 
to a moving turbine there are many considerations that needs to be made. Allthough the equations have been known for over 200 years no one 
has been able to prove or disprove the well-posedness of the problem. FINAL ELEMENT FORMULATION WELL-POSED?? Some of the most common algorithms 
will be discussed in the following subsections

\begin{itemize}
\item coupled versus decoupled 
\item The Uzawa algorithm 
\item Pressure correction method
\item fractional step method
\end{itemize}

\section{nek5000}
Nek5000 is a turbulent flow solver developed mainly by Paul Fischer and has through the past 20 years had several contributours. 
It is an open-source code applicable to many different types of flow and it has been put a lot of effort into making the code as 
parallelizable as possible. With SEM as the numerical method applied it is possible to obtain extremely accurate results.  

nek5000 has its own mesh-generator for generating simpler geometries and it is also implemented the possibility to integrate CUBIT mesh 
files. A common starting point for simulating turbulent flow is a geometry given by a CAD-file, without any functions to describe the 
boundaries. The possibility to make a mesh through a visual gui is provided through for instance ICEM. This mesh can then later be transformed
to the inputfile required by nek5000. 

So far nek5000 has supported three automatic routine for generating curved edges; circles in 2-D geometries, spherical shell elements and 
a general 2nd degree interpolation. Further manipulation of the element edges is left to the user to define manually for each particular 
problem. 

\begin{itemize}
\item Application areas
\item Description of the numerical algorithm 
\item usage of the program 
\end{itemize}

