% Chapter 2 - physics theory

\chapter{Fluid dynamics} % Main chapter title

\label{physics} % Change X to a consecutive number; for referencing this chapter elsewhere, use \ref{ChapterX}

\lhead{Chapter 2. \emph{background on fluid dynamics}} % Change X to a consecutive number; this is for the header on each page - perhaps a shortened title

%----------------------------------------------------------------------------------------
%	SECTION 1
%----------------------------------------------------------------------------------------
\section{Navier-Stokes equation}

The physics regarding fluids in motion are described mathematically by the Navier-Stokes equation. The equations can be derived in many ways, 
and it is referred to \cite{White} for a complete explication. The general idea is to conserve momentum and mass in a control domain and by 
applying Reynolds transport theorem and the result is the following two equations reffered to as the N-S equations. 
\begin{itemize}
\item N-S eq.
\end{itemize}
These equations have been studied for centuries and different physical situations lead to different simplifications and sets of equations. 
Examples are the Euler equations, Stokes problem, darcy flow and the incompressible navier-stokes equations. 
Without doing the derivation of the equations a short discussion of each term in equation N-S will be given here in order to understand there
physical origin and their mathematical attributes. 

\begin{itemize}
\item The time-derivative
\item The convective term 
\item The diffusive term 
\item The pressure gradient
\item The reynolds stress tensor
\item The reynolds number
\item The mass equation
\end{itemize}


\section{Resolving the turbulent term}
Depending on the wanted accuracy of your solution and the number of computational units at hand the turbulent term can be solved in 
different ways. A brief overview of the different approaches to the turbulent term are discussed in the following subsections

\begin{itemize}
	\item Laminar flow
	\item RANS
	\item LES
	\item K-epsilon and K-omega
	\item DNS
\end{itemize}

\section{Solution methods of incombressible N-S}
A non-linear set of equations requires a non-trivial solution method, and when the domain of the problem can be anyting from a simple channel 
to a moving turbine there are many considerations that needs to be made. Allthough the equations have been known for over 200 years no one 
has been able to prove or disprove the well-posedness of the problem. FINAL ELEMENT FORMULATION WELL-POSED?? Some of the most common algorithms 
will be discussed in the following subsections

\begin{itemize}
\item coupled versus decoupled 
\item The Uzawa algorithm 
\item Pressure correction method
\item fractional step method
\end{itemize}

\section{nek5000}
Nek5000 is a turbulent flow solver developed mainly by Paul Fischer and has through the past 20 years had several contributours. 
It is a open-source code applicable to many different flow situations.   

\begin{itemize}
\item Application areas
\item Description of the numerical algorithm 
\item usage of the program 
\end{itemize}

