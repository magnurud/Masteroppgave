% Chapter 2 - physics theory

\chapter{Problem description} % Main chapter title

\label{physics} % Change X to a consecutive number; for referencing this chapter elsewhere, use \ref{ChapterX}

\lhead{Chapter 2. \emph{background on fluid dynamics}} % Change X to a consecutive number; this is for the header on each page - perhaps a shortened title

%----------------------------------------------------------------------------------------
%	SECTION 1
%----------------------------------------------------------------------------------------
\section{The Incompressible Navier-Stokes equation}

The physics regarding fluids in motion are described mathematically by the Navier-Stokes equation. 
The equations can be derived in many ways, and it is referred to \cite{White} for a complete derivation.
The general idea is to conserve momentum and mass in a control domain providing a system of two equations.
In this thesis only the incompressible N-S equation will be considered, 
with the assumption of an incompressible flow the conservation of mass 
results in a divergence free restriction on the solution $u$.
Without further introduction the non-dimensional incompressible  N-S equations are stated as  
%
\begin{align}
    \begin{split}
    \frac{\partial \mathbf{u}}{\partial t} + \mathbf{u}\cdot \nabla\mathbf{u} &= -\nabla p + \frac{1}{Re}\nabla \cdot \tau, \\
		\nabla \cdot \mathbf{u} &= 0.
    \end{split}
	\label{eq:NS}
\end{align}
%
These equations have been studied for centuries and different physical situations
lead to different simplifications and sets of equations.
Examples are the Euler equations,
Stokes problem and darcy flow.
In particular the Stokes problem is applied a lot as an initial test of the full N-S problem. 
Before attempting to solve these equations it is important to understand the role of each term 
and their mathematical influence on the problem. 
\begin{itemize}
    \item $\partial \mathbf{u} /\partial t$
     - The time-derivative of the flow, for a steady state flow this term will be equal zero.
             The discretization of this term is often based on some implicit scheme in order to improve stability.  
    \item $\mathbf{u} \cdot \nabla \mathbf{u}$
     - The convective term, describes the transport due to the flow itself on each of its components. 
    The term is not present in Stokes problem.
    The mathematical operator corresponding to this term is non-linear and non-symmetric, and does therefore require the equations to be solved 
    by some iteration procedure. Even with linear advection the operator is a source to instability and needs to be handled carefully. 
\item  $\nabla \cdot \tau$ 
       - The Reynolds stress tensor, simplifies to $ \Delta \mathbf{u}$ in laminar flows but is treated as 
       $\nabla \cdot [ \nabla \mathbf{u} + \nabla^T \mathbf{u}] $ in turbulent flows and introduces 
       6 new unknows, resolving this tensor is known as the closure problem. $\nu$ is the viscosity of the fluid.
    \item $\nu \Delta \mathbf{u}$ 
    -The diffusive term describes the natural diffusion of the fluid. The effect of diffusion is determined by the 
    viscosity of the fluid. The corresponding operator stabilizes the the system and it is therefore generally easier
    to solve the N-S equations for high-viscosity fluids. 
    \item $\nabla p$
    - The pressure gradient, Removal of this term results in a pure advection diffusion problem.
    \item $\nabla \cdot \mathbf{u}$ 
    - The divergence free condition from the mass equation.
    \item $Re$ 
    - The Reynolds number defined as $UL/\nu$ where $U$ is the velocity scale, $L$ is the length scale and $\nu$ 
      is the viscosity of the fluid. $Re$ describes the relation between the biggest length scales of the flow
      and the viscous length scales. Notice how for large Reynolds number the unstable non-linear term 
      dominates the transportation.Turbulent flows are characterized by high reynolds number.
\end{itemize}
For large Reynolds number the huge span in length scales requires an extremely fine mesh if the equations \ref{eq:NS} 
are to be solved exactly. Because a fine mesh implies a high computational cost a direct numerical solution (DNS) is not feasible for 
problems of a certain geometrical complexity. There are many different approaches on how to resolve the turbulent term and in 
this thesis the main approach will be through Large Eddy Simulations (LES) which will be discussed 
in the following section.
%
\subsection{Weak formulation of N-S}
The numerical algorithms applied in this thesis require a weak formulation of equation~\ref{eq:NS}.
Before the weak form is derivated a few operators will be defined in order to simplify the final 
expression.
\begin{align}
    ( \cdot, \cdot)_{L_2} &= \int_{\Omega}\cdot\cdot d\Omega\\
    \mathcal{A}(\mathbf{u},\mathbf{v}) &= (\nabla\cdot\mathbf{u},\nabla\cdot\mathbf{v})_{L_2}\\
    \mathcal{B}(\mathbf{u},p) &= (\nabla \cdot \mathbf{u},p)_{L_2}\\
    \mathcal{C}(\mathbf{w};\mathbf{u},\mathbf{v}) &= (\mathbf{w}\cdot \nabla \mathbf{u},\mathbf{v})_{L_2}\\
    \label{eq:weakoperators}
\end{align}
A weak formulation is obtained by multiplying with a test funcion $\mathbf{v}$ and integrating over
the entire domain.
\begin{align}
    \begin{split}
        \int_{\Omega}\frac{\partial \mathbf{u}}{\partial t}\cdot\mathbf{v}d\Omega
        + \int_{\Omega}\mathbf{u}\cdot \nabla\mathbf{u}\cdot\mathbf{v}d\Omega
        &= -\int_{\Omega}\nabla p\cdot \mathbf{q} d\Omega 
        + -\int_{\Omega}\frac{1}{Re}\nabla \cdot \tau\cdot\mathbf{v}d\Omega, \\
		\int_{\Omega}\nabla \cdot \mathbf{u}\cdot\mathbf{v}d\Omega &= 0.
    \end{split}
	\label{eq:NSweak}
\end{align}
\begin{align}
    \begin{split}
        (\frac{\partial \mathbf{u}}{\partial t},\mathbf{v})
        + (\mathbf{u}\cdot \nabla\mathbf{u},\mathbf{v})
        &= -(\nabla p, \mathbf{v} ) 
        -(\frac{1}{Re}\nabla \cdot \tau,\mathbf{v}), \\
		(\nabla \cdot \mathbf{u},p) &= 0.
    \end{split}
	\label{eq:NSweak}
\end{align}
Writing the covective term in conservational form $\nabla \cdot \mathbf{u}\mathbf{u}$ and applying the 
divergence theorem on the terms leaves us with 
\begin{align}
    \begin{split}
        (\frac{\partial \mathbf{u}}{\partial t},\mathbf{v})
        - (\mathbf{u}\mathbf{u},\nabla \cdot \mathbf{v})
        &= +(p,\nabla \cdot \mathbf{v} ) 
        +\frac{1}{Re}(\tau,\nabla \cdot \mathbf{v}), \\
		-(\mathbf{u},\nabla p) &= 0.
    \end{split}
	\label{eq:NSweak}
\end{align}
Finally, by using the notation introduced in ~\ref{eq:weakoperators} the weak formulation of the incompressible
N-S equation can be stated as 
\begin{align}
    \begin{split}
        (\frac{\partial \mathbf{u}}{\partial t},\mathbf{v})
        - \mathcal{C}(\mathbf{u};\mathbf{u},\mathbf{v})
        &= \mathcal{B}(\mathbf{v},p) 
        +\frac{1}{Re}\mathcal{A}(\mathbf{u},\mathbf{v}), \\
        -\mathcal{B}(\mathbf{u},p) &= 0.
    \end{split}
	\label{eq:NSweak}
\end{align}


%
\section{The passive scalar equation}
The N-S equation explains how a fluid will behave and solving it provides a complete pressure-velocity field of the 
domain of interest, but it does not provide the answer of how heat will distribute in this flow, or how a gas will spread.
The equation corresponding to this physical problem will be reffered to as the passive scalar equation for any scalar 
$\phi_i$ and is stated as 
\begin{align}
    \begin{split}
        (\rho c_p)_i\frac{\partial \phi_i}{\partial t} + \mathbf{u}\cdot \nabla\phi_i 
        &= \nabla \cdot(k_i\nabla \phi_i)+ (q_{vol})_i.
    \end{split}
	\label{eq:PS}
\end{align}
\colorbox{yellow}{How to explain the constant in this equation ??} 
\section{Resolving the turbulent term using LES}
When DNS is not feasible due to high Reynoldsnumber LES is one of the most powerful tools when it comes to simulating turbulent flows.
The idea is based on the fact that for the small turbulent structures behave independently of time and location and are therefore 
easy to model. This way the larger structures driven by geometry, inflow conditions and external forces can be simulated using a coarser 
grid while the effect of the small structures are modelled. \colorbox{yellow}{Some source on the importance of SGS-model}

In order to perform a LES a filter width a certain width $\Delta$ has to be chosen, 
the filtered N-S equations are given as 
%
\begin{align}
    \begin{split}
        \frac{\partial \mathbf{\bar{u}}}{\partial t} + \mathbf{\bar{u}}\cdot \nabla\mathbf{\bar{u}}
        &= -\nabla \bar{p} +\nu\Delta \mathbf{\bar{u}}-\nabla \cdot \tau, \\
        \nabla \cdot \mathbf{\bar{u}} &= 0.
    \end{split}
	\label{eq:NSfiltered}
\end{align}
%
Where $\tau$ in this case denotes the subgrid scale (SGS) stress given as 
\begin{align}
    \tau_{ij} = \overline{u_iu_j}-\overline{u}_i \overline{u}_j
    \label{eq:sgstensor}
\end{align}

The problem is now reduced to modelling this tensor, the most common SGS-model is the 
dynamic Smagorinsky-Lilly model.
\subsection{dynamic Smagorinsky-Lilly SGS-model}

\section{Solution methods of incombressible N-S}
A non-linear set of equations requires a non-trivial solution method, and when the domain of the problem can be anyting from a simple channel 
to a moving turbine there are many considerations that needs to be made. Allthough the equations have been known for over 200 years no one 
has been able to prove or disprove the well-posedness of the problem. FINAL ELEMENT FORMULATION WELL-POSED?? Some of the most common algorithms 
will be discussed in the following subsections

\subsection{Operator-splitting techniques }
Let us considere a simplified transient problem 
\begin{align}
    \frac{du}{dt} = f + g.
    \label{eq:testproblem}
\end{align}
both $f$ and $g$ are functions of $u$ and therefore of $t$ as well. By integrating on both sides one obtains
\begin{align}
    u(s) - u(0) = \int_0^s f dt + \int_0^s g dt.
    \label{eq:testproblemintegrated}
\end{align}
The idea is then to evaluate one of the integrals by an explicit scheme and the other one with an implicit scheme.
A general formulation of this for a timestep would be on the form 
\begin{align}
    u^{n+1}-u^{n} = \sum_{k = 0}^{N} a_k f^{n-k}+\sum_{k = 0}^{N} b_k g^{n-k+1}.
    \label{eq:exp-imp}
\end{align}
In equation~\ref{eq:exp-imp} the first integral in ~\ref{eq:testproblemintegrated} is evaluated by 
an explicit scheme and the second one by an implicit scheme. The theoretical foundation for 
this method is rather technical and the reader is refered to Maday et al [262] for further details.

\colorbox{yellow}{Maday,Ronquist An operator-integration-factor splitting method for time dependent problems.}

%Now introducing an integrational factor $Q(t)$ the initial problem ~\ref{eq:testproblem} can be written as


Making a efficient and stable numerical scheme requires some intelligent splitting technique. 
OIFS allows one to treat the diffusive term implisitly while the convective term explicitly.
By considering the NS-equation in a general operational form 
\begin{align}
    M\frac{d v_i}{dt} + Cv_i = -Kv_i +D_i p +Mf_i.
    \label{eq:NSoperator}
\end{align}
Now let us define an operator $Q(t)$ such that 

\begin{align}
    \frac{dQ(t)M\mathbf{v}}{dt} &=  Q(t)M\frac{d\mathbf{v}}{dt} + \frac{d}{dt}(Q(t)M)\mathbf{v}.\\
    &= M\frac{d\mathbf{v}}{dt} + C\mathbf{v} \text{  when t = t*. }
    \label{eq:integrationalfactor}
\end{align}
%
This way equation~\ref{eq:NSoperator} can be written as 
\begin{align}
    \frac{d QM\mathbf{v}}{dt} =Q( -K\mathbf{v} +D p +M\mathbf{f}.)
    \label{eq:NSoperatorOIFS}
\end{align}

\subsection{Fractional step}
Fractional step is an algorithm that can be divided into four seperate steps. For simplicity let us write the N-S
equation as 
\begin{align}
    \partial_t v = Lv + Dp + Nv.
    \label{eq:NSfracstep}
\end{align}
Where $\partial_t, L,D,P$ Denotes the time-derivate, linear, gradient and non-linear operator. 
The method describing one time-step can then be summarized as following
\begin{itemize}
    \item solve $v^* = v^n + \Delta t Nv^n$.
    \item solve the poisson pressure equation $\Delta\: p = \nabla \cdot (v^*/\Delta t)$.
    \item solve $v^{**} =v^* + \Delta t Dp^{n+1}$.
    \item solve $\partial_t v^{n+1}=v^{**} + \Delta t Lv^{n+1}$.
\end{itemize}
This method provides an efficient algorithm, but is known to produce errors of order O(1).
The problem is the pressure poisson equation which is solved with incorrect boundary 
conditions.
\subsection{Pressure-correction}

%\subsection{Decoupling the pressure} 

%A common way to approach equation~\ref{eq:NS} is to solve the pressure and velocity field
%seperatly. The mathematical reasoning for this approach is obtained by taking the 
%divergence on both sides, 
%\begin{align}
	%\nabla \cdot \frac{\partial \mathbf{u}}{\partial t} + 
    %\nabla \cdot(\mathbf{u} \cdot \nabla) \mathbf{u} 
    %&= -\Delta p + \nu \Delta \tau \\
	%\frac{\partial (\nabla \cdot  \mathbf{u})}{\partial t} + 
    %(\nabla \cdot \mathbf{u}) \cdot \nabla \mathbf{u} 
    %&= -\Delta p + \nu \Delta \tau \\
   %\Delta p &= \nu \Delta \tau \\
	%\label{eq:pressuredecoupling}
%\end{align}

%Where the divergence free criterium is applied to get to the final equation.
