% Chapter 5

\chapter{Results} % Main chapter title

\label{results} % For referencing the chapter elsewhere, use \ref{Chapter1} 

\lhead{Chapter 5. \emph{Results and Discussion}} % This is for the header on each page - perhaps a shortened title

%----------------------------------------------------------------------------------------

\section{Drag and lift on a cylinder}
The effect of the implemented algorithm in Nek5000 explained in Chapter~\ref{implementation} is
illustrated by solving a laminar flow test problem. 
The solution is compared with previously benchmark computations performed by a number of 
contributors~\cite{benchmark}. The test problem considered is steady flow with Re=20 in 
a rectangular channel past a cylinder. The drag and lift coefficients on the cylinder 
are calculated and compared to a pair of reference values. The results can be found in 
table~\ref{tab:testcase}. As the results clearly show the treatment of the geometry is 
crucial, both coefficients are computed with significantly better accuracy. 
%
\begin{table}
\centering
\begin{tabular}{l l c c c c}
		\toprule
		\# of Cells & Software & $c_D$ & $c_L$ & \%\textbf{Err} $c_D$ &\%\textbf{Err} $c_L$ \\ \midrule 
		2070 & Nek5000 (mid) & 6.18349 & 0.008939 & 0.030 & 4.19 \\ 
		2070 & Nek5000 (arc) & 6.18498 & 0.009413 & 0.006 & 0.13 \\
		3145728 & CFX 		 & 6.18287 & 0.009387 & 0.04 &0.15 \\
		3145728 & OF	     & 6.18931 & 0.00973 & 0.06 &3.5 \\
		3145728 & FEATFLOW   & 6.18465 & 0.009397 & 0.01 &0.05 \\
		\bottomrule	
	\end{tabular}
	\caption{Results for the drag and lift coefficients with refences values 
	$c_D = 6.18533$ and $c_L = 0.009401$.}
\label{tab:testcase}
\end{table}
%
The polynomial degree used in the calculations with Nek5000 was chosen as $p = 11 $ 
in all directions. The explicit number of cells is therefore $2070\cdot11^{3} = 2755170$,
approximately $88\%$ of the number of cells used for the benchmark simulations. Compared 
with the results from the other softwares applied in~\cite{benchmark} Nek5000 performs 
just as well or better in most cases. It should be mentioned that the division of the grid is done 
differently for Nek5000 so the comparison is not as direct as it may seem from the table.
\colorbox{yellow}{add more info on the mesh??}

%----------------------------------------------------------------------------------------
\section{Gas dispersion in a simplified urban area} 
Results to come \ldots 

\section{Discussion and Conclusion}
\colorbox{yellow}{How did Nek perform overall, user-friendly ?,correctness,speed etc.}

