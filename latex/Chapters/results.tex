% Chapter 6

\chapter{Case studies and results} % Main chapter title

\label{results} % For referencing the chapter elsewhere, use \ref{Chapter1} 

\lhead{Chapter 6. \emph{Results and Discussion}} % This is for the header on each page - perhaps a shortened title

%----------------------------------------------------------------------------------------

This chapter will present the cases investigated and the results acquired.
The main tools in addition to Nek5000 used to obtain the results in this 
chapter are \textit{ANSYS ICEM} and \textit{python}. For post-processing \textit{Visit} and
\textit{Matlab} were used.
%
\section{Case 1: Gas dispersion in a simplified urban area} \label{case1}
%The problem investigated in this work is gas dispersion of neutral gas in a velocity field through four cubic blocks.
%Similar simulations have been done in CDP and Fluent which are compared to data from a wind-tunnel experiment performed by ALAN.
The scenario investigated in this case is dispersion of a neutral gas in a rectangular tunnel
with four cubic blocks placed as obstacles. The blocks have sides $h = 0.109$m and represent a 
set of buildings forming a street canyon. The gas is released from a circular source on 
ground level and
is translated by the wind field through the canyon, see \fref{fig:layout}.
In this figure $h$ has been used as the length scale. The dotted lines
indicate the positions where data is collected.
%
\begin{figure}[h]
	\includegraphics[width=1.1\textwidth]{Figures/layout.eps}
	\caption{Schematic overview of the domain from above. The data is collected along the dotted lines.}
	\label{fig:layout}
\end{figure}
%

Scaling the entire domain with the size of the boundary layer $H =1$m restricts it to
the box $0.0\leq x/H \leq 4.96,-1.75\leq y/H \leq 1.75, 0\leq z/H \leq 1.5$.
The four cubic boxes are centered around $(1.4315,0)$ with a distance $h$ between each box.
The source is placed with its center in $(0.396,0)$ and has a radius $r = 0.0515$.
The grid used for the computations consists of 14747 elements and a polynomial degree of
7, this makes the total number of nodes $N\approx 5.2$mill. There are no curved elements in this case, so the 
circular source is created simply by requiring that the inflow velocity is non-zero only for the nodes
on the floor within that particular circle. The corresponding area is somewhat different from that
of a circle $\approx 5 \%$, but this is scaled correctly during the post-processing.

The simulations are performed using Large Eddy Simulation (LES) 
with the dynamic Smagorinsky-Lilly subgrid-scale model and by applying the polynomial filtering
routine that is available in Nek5000. 
The release of gas will result in a plume that is advected with the wind field,
see \fref{fig:plume}. The concentration of the released gas at the 
indicated positions in \fref{fig:layout} are compared with 
experimental wind-tunnel data and simulations performed in CDP~\cite{CDP}. 
For clarification some of the variables repeatedly mentioned in this section are
stated explicitly in \tref{tab:simplevariables}.
\begin{table}[h]
    \centering
    \begin{tabular}{c c c c}
        Variable & value & unit & commentary \\ \hline
        $H$   & $1$ & m & length scale of the domain \\ 
        $h$   & $0.109$ & m & the sides of the cubic boxes\\ 
        $Q$   & $50$ & dm$^3$/min & gas release from source \\ 
        $U_{ref} $& $\approx1.08$ & m/s & reference value of $U$ \\
    \end{tabular}
    \caption{Essential variables, $U_{ref}$ is calculated as a time average of the velocity in 
        x-direction at a point far away from the floor and walls and will therefore 
        vary by a small amount from case to case. }
    \label{tab:simplevariables}
\end{table}

The inflow condition at $x=0$ has to be mapped onto the boundary at each time step. To mimic the situation in the wind-tunnel the velocity field
on the inflow was generated in a different simulation performed in CDP. The inflow velocity was written to file every 
$0.0013$s for a total of $28$s and had to be interpolated onto the domain for the simulations in Nek5000 since the grid was not identical.
The right plot in \fref{fig:inflow} is an instantaneous picture of the inflow velocity in x-direction, 
notice how the pattern repeats itself along the y-axis. This is because the inflow data was generated in a smaller channel, approximately $1/3$ of the 
width of the computational domain used for the data sampling. An interpolation algorithm implemented at FFI was applied to adjust 
the inflow data to the computational mesh, this was done directly in \verb|.usr|. 
%
\begin{figure}
\centering
  \centerline{
\begin{minipage}{.6\textwidth}
  \includegraphics[width=1.0\linewidth]{Figures/inflow_field_avg.png}
  %\captionof{figure}{A figure}
\end{minipage}%
\begin{minipage}{.6\textwidth}
  \centering
  \includegraphics[width=1.0\linewidth]{Figures/inflow_field.png}
\end{minipage}
}
  \caption{The averaged (left) and instantaneous (right) x-velocity on the inflow boundary.}
  \label{fig:inflow}
\end{figure}
%

%
\begin{figure}[h]
	\centering
	\includegraphics[width=0.6\textwidth]{Figures/plume2.png}
	\caption{An iso-surface of the average concentration with $C=0.03$ 
    after 22 seconds of sampling. Streamwise direction is from left to right.}
	\label{fig:plume}
\end{figure}
%
The simulations in Nek5000 were performed in the following manner; first 6 seconds of flow-through to initialize the velocity field 
in the channel, followed by 8 seconds of gas release to initialize the gas concentration. After assuring that the wind-field was 
correctly created and the released gas had reached the measurement lines furthest from the source, the data sampling of 22 seconds 
started.

The mesh used in the simulations performed in Nek5000 and the one performed in CDP are 
different, and the resolution in the part of the domain close to the cubes is described 
in~\tref{tab:meshdiff}.
It is evident that the resolution is better in the simulations done in CDP and especially in 
the $z$-direction. 
\begin{table}[h]
    \centering
    \begin{tabular}{c| c c c}
        Solver   & $n_x$& $n_y$ & $n_z$ \\ \hline
        CDP      & 28 & 28 & 64 \\ 
        Nek5000  & 22 & 22 & 36 
    \end{tabular}
    \caption{Number of nodes used to represent one cube.}
    \label{tab:meshdiff}
\end{table}


\subsection{Results - Gas dispersion} 
This case is a part of a larger project designed to evaluate different solvers 
ability to perform simulations of gas dispersion. The N-S equations are solved using
the $P_NP_N$ formulation with the fractional step method, IOFS with a target CFL-number 
equal 2.0 was enabled to maximize the time step as recommended in~\cite{Nek}. It should be 
mentioned that when activating the SGS model and deactivating the 
filter the simulation is more probable to blow up. This effect is captured in~\fref{fig:maxvel}
that shows how the Smagorinsky model does not damp spurious velocity modes in the 
same degree as the filter. 

%
\begin{figure}[h]
	\centering
	\includegraphics[trim=0.5cm 7cm 0.5cm 7cm, width=0.6\textwidth]{Figures/maxvel.pdf}
    \caption{$||\mathbf{u}||_{\infty}$ as a function of time. The green line represents the 
simulation with the dynamic Smagorinsky SGS model and the blue line represents the filtering 
with $\alpha = 0.05$ and a quadratic decay on the last 3 modes.}
	\label{fig:maxvel}
\end{figure}
%
%
%\begin{figure}[h]
	%\centering
	%\includegraphics[width=0.8\textwidth]{Figures/NekcH.pdf}
	%\caption{Time-averaged concentration with a sample time of $18.00$ s at $z/H = 0.025$ plotted horizontally and scaled 
	%with the free-stream velocity and emission rate. Compared against wind tunnel data.
%Two dashed lines on either side of the centerline represent the canyon.}
	%\label{fig:cHfilter}
%\end{figure}
%
%
%\begin{figure}[h]
	%\centerline{\includegraphics[width=0.8\textwidth]{Figures/vel_field.png}}
	%\caption{velocity field for $z= 0.02$m, around the source and the cubes.}
	%\label{fig:vel_field}
%\end{figure}
%

%\colorbox{green}{redo these simulation in case they were started to early.}
%
%\begin{figure}[h]
	%\centering
	%\includegraphics[width=0.8\textwidth]{Figures/NekcV.pdf}
	%\caption{Time-averaged concentration with a sample time of $18.00$ s at $y = 0$ plotted
    %vertically and scaled 
	%with the free-stream velocity and emission rate. Compared against wind tunnel data.
%Two dashed lines on either side of the centerline represent the canyon.}
	%\label{fig:cVfilter}
%\end{figure}
%
%
%\begin{figure}[h]
	%\centering
	%\includegraphics[width=0.8\textwidth]{Figures/Nek_smag_cV.pdf}
	%\caption{Time-averaged concentration with a sample time of $22.00$ s at $y = 0$ plotted
    %vertically and scaled 
	%with the free-stream velocity and emission rate. Compared against wind tunnel data.
%Two dashed lines on either side of the centerline represent the canyon.}
	%\label{fig:cVsmag}
%\end{figure}
%
%
%\begin{figure}[h]
	%\centering
	%\includegraphics[width=0.8\textwidth]{Figures/Nek_smag_cH.pdf}
	%\caption{Time-averaged concentration with a sample time of $22.00$ s at $y = 0$ plotted
    %vertically and scaled 
	%with the free-stream velocity and emission rate. Compared against wind tunnel data.
%Two dashed lines on either side of the centerline represent the canyon.}
	%\label{fig:cVsmag}
%\end{figure}
%
\newpage
\begin{figure}[h]
    \centering
    \includegraphics[width=0.9\textwidth]{Figures/NekcH_all.pdf}
    \caption{Time-averaged concentration with a sample time of $22.00$s at $z/H = 0.025$ plotted
    and scaled with the free-stream velocity and the emission rate. Compared with wind-tunnel data 
    and results from CDP. Two dashed lines on either side of the centerline represent the canyon.}
    \label{fig:cHall}
\end{figure}

\fref{fig:cHall} shows the scaled concentration along the dotted lines in~\fref{fig:layout}. 
According to this figure Nek5000 does indeed capture the important features of the mean concentration.
At the two first measurement lines the results are slightly skewed to the right, this is to some degree 
also the case for the CDP simulations but not for the experiment. A possible explanation could be that 
the inflow condition favours one of the sides of the domain, or simply that the sampling time is not 
sufficiently long. 
%Along the second measurement line which is placed in the middle of the 4 cubes Nek5000 
%estimates a concentration peak lower than both the reference solutions.  

The results also indicate that the difference between the SGS model and the filtering
is not that large,
if anything the SGS model shows a tendency to estimate higher concentration peaks.
In particular the first plot indicates a significant difference.
An important difference between the filtering and the SGS model is 
that the filter works based on the current state of the flow 
whereas the amount of diffusion added by the SGS model is strongly biased 
by the previous states of the flow.
This could lead to local observations of either too much or too little smoothing.

The concentration along the vertical measurement lines is plotted in \fref{fig:cVall} and overall 
Nek5000 provides good results according to the reference solutions. The largest difference is found close 
to the wall right in the middle of the cubes. In particular the simulation with filtering 
underestimates the concentration in this domain. The resolution of the mesh used for the 
Nek5000 simulations in this area is notably worse than for the CDP-simulations.
And in the middle of the cubes neither one of the filter or the Dynamic Smagorinsky model
are able to correct this.
The $P_NP_N$ formulation is known to produce splitting errors of significant sizes 
close to the wall, and could play an important role in this part of the domain.

\begin{figure}[h]
    \centering
    \includegraphics[width=0.95\textwidth]{Figures/NekcV_all.pdf}
    \caption{Time-averaged concentration with a sample time of $22.00$s at $y = 0$ plotted
    and scaled with the free-stream velocity and the emission rate. Compared with wind-tunnel data
    and results from CDP.}
    \label{fig:cVall}
\end{figure}
The simulations were also performed using the $P_NP_{N-2}$ formulation with filtering, and the 
results were similar to those observed above. The SGS model did however not function at all for 
the $P_NP_{N-2}$ implementation, and resulted in a system crash in one of the earliest time steps.
It was beyond the scope of this work to correct this presumable bug in the Nek5000 implementation. 

It was also experimented with different meshes. An attempt was made to improve the 
results at the measurement lines in between the cubes by increasing the resolution in
this particular area, this was done by using the refinement functionality in ICEM.
The rest of the mesh was made coarser to keep the amount of elements at approximately the 
same level. The amount of nodes used to describe one cube as presented in~\tref{tab:meshdiff} 
were for this simulation $(n_x,n_y,n_z) = (21,21,63)$.
The simulation was performed using $P_NP_N$ with filtering, but the results were 
similar to the ones previously obtained. It could seem like a refinement of the 
mesh in the $x$ and $y$ direction could be necessary to achieve better results. 

\newpage
As for the performance of Nek5000 the results are baffling. With the same number of processors,
same accuracy criteria, and approximately the same number of nodes in the grid, Nek5000 simulated
one second of flow more than five times as fast as a numerical code comparable to CDP! 
This was done without any particular tuning and with a time step about $2/3$ of the one used in 
CDP. 

To check how Nek5000 scales for the number of processors, one second of flow was 
simulated using the $P_NP_{N-2}$ formulation on the grid with 14747 elements,
$N=7$, $DT = 0.001$ and both IOFS and filtering was activated. The results 
are presented in~\fref{fig:perfgas}.
%
\begin{figure}[h!]
	\centerline{
        \includegraphics[trim=0.5cm 4cm 0.5cm 4cm, width=0.6\textwidth]{Figures/scalresults.pdf}}
        \caption{Scaling results, from case 1 with $14747$ elements, $P_NP_{N-2}$ formulation using filtering and $N=7$.}
	\label{fig:perfgas}
\end{figure}
%

This is far from optimal speedup which is the theoretical upper limit, and Nek5000 were perhaps expected to perform 
better than these results indicate. This performance test is done without doing any tweaking of the settings, and 
it is probable that by adjusting some parameters in \verb|.rea| a better speedup could be obtained. 

%\begin{figure}[h]
    %\centering
    %\includegraphics[width=0.8\textwidth]{Figures/Nek_smag_cfluctH.pdf}
    %\caption{Time-averaged concentration with a sample time of $22.00$ s at $y = 0$ plotted
    %vertically and scaled 
    %with the free-stream velocity and emission rate. Compared against wind tunnel data.
%Two dashed lines on either side of the centerline represent the canyon.}
    %\label{fig:cVsmag}
%\end{figure}


\section{Case 2: Drag and lift on a cylinder}
A standard benchmark case for flow solvers is presented in~\cite{benchmark}. 
The case is to calculate the drag and lift coefficients on a cylinder in a rectangular channel.
The setup for the domain and boundary conditions are given in \fref{fig:cylinder}.
The constants applied in the description of the geometry and the coefficient scales are listed 
in table \tref{tab:case2consts}.
%
\begin{table}[h]
    \centering
    \begin{tabular}{c l l}
     Constant & Value & Property \\ \hline
    $H$ & $0.41\text{m}$ & Width and height for the channel \\
    $D$ & $0.1\text{m}$ & Diameter of the cylinder and length scale \\
    $U$ & $0.2\text{m/s}$ & Velocity scale \\
    $\nu$ &  $ 10^{-3}\text{m$^2$/s}$ & Kinematic viscosity of the fluid \\
    $Re$ & $20$ & Reynolds number \\ 
    \end{tabular}
    \caption{Constants for case 2}
    \label{tab:case2consts}
\end{table}
%
Finding the drag and lift coefficient requires a calculation of the velocity field around the cylinder
which is done by solving the unsteady N-S equations until a steady flow is reached. This implies that the
spatial accuracy will dominate the error and one would expect great results in Nek5000 
due to its spectral convergence rate.

The flow is laminar with Reynolds number $Re=20$ so all the 
challenges arising when dealing with turbulent flow does not come to play in this case. 
%
\begin{figure}[h]
    \centering
    \includegraphics[width = 1.0\textwidth]{Figures/cylinder.pdf}
    \caption{Computational domain and boundary conditions.}
    \label{fig:cylinder}
\end{figure}
%
The drag and lift forces on a surface $S$ are given as 
%
\begin{align}
    F_D = \int_{S}(\rho \nu \frac{\partial v_t}{\partial n}n_y-pn_x)dS, \qquad
    F_L = -\int_{S}(\rho \nu \frac{\partial v_t}{\partial n}n_x+pn_y)dS.
    \label{eq:dragnlift}
\end{align}
%
$v_t$ is the tangential velocity, $\mathbf{n}=[n_x,n_y,0]$ is the unit vector normal to the surface $S$ 
and the tangent velocity vector is defined as $\mathbf{t} = [n_y,-n_x,0]$.
 
Surface integrals in Nek5000 are solved numerically, $\int_S f dS = \sum f_i A_i$, where $f$ is some function and 
$A_i$ is the quadrature weight to the $i$th GLL-node for the surface integral over face S.
$A_i$ corresponds to a two dimensional mass matrix in Nek5000 available for all elements and all faces in the array 
\verb|area(lx1,lz1,6,lelt)|.

The coefficients corresponding to these forces known as the drag and lift coefficients 
are given by the formulas 
\begin{align}
    c_D = \frac{2F_D}{\rho U^2 D H}, \qquad
    c_L = \frac{2F_L}{\rho U^2 D H}.
    \label{eq:dragnliftcoeffs}
\end{align}


Nek5000 provides functions for calculating lift and drag on any user-specified object.
The function is called \verb|drag_calc(scale)|, with the input parameter 
defined by the user, for this case \verb|scale| $=2/(\rho U^2DH)$.  
Apart from this the function \verb|set_obj()| has to be modified to create an object 
that consists of pointers to all the faces on the cylinder.
%Let $x,y$ be points in the computational domain, $x_c,y_c$ be the coordinates to the 
%center line in the cylinder and $0<tol\ll1$ be some user defined tolerance. The faces that belong to the cylinder can then be found by 
%looping over all elements and their faces evaluating $\epsilon = \sqrt{(x-x_c)^2+(y-y_c)}$.
%If $\epsilon < tol$ for an entire face then this face is known to 
%belong to the cylinder and is added to the object. Nek also allows the user to specify multiple objects 
%assigning the faces of interest to object 1, object 2 etc. The geometry and mesh 
%for this case was generated in ICEM, and the total number of elements are 2070. 
%For the final calculation polynomial degree $P = 11$ was applied leading to a 
%total of $N = 2755170$.
%
\begin{figure}[h]
    \centering
    \includegraphics[trim=0.5cm 4cm 0.5cm 4cm,width = 0.3\textwidth]{Figures/cyl_elem.pdf}
    \caption{Initial mesh around cylinder.}
    \label{fig:cyl_elem}
\end{figure}
%
The mesh around the cylinder is illustrated in \fref{fig:cyl_elem}.
Initially this case was solved using a second degree polynomial to describe the circle segments
corresponding to each element. With the implementation of the new routine as described 
in~\cref{xyzarc} the circle segments could be represented with the same order as 
the polynomials used for the velocity. The importance of the error resulting 
from the second degree approximation of the circle segments is presented in \cref{results}.
%Note that these elements was split in three, in order to obtain 
%a finer mesh in the region of interest. Of the elements numbered in~\fref{fig:cyl_elem}
%only the first six contains edges on the cylinder.Hence the second order polynomials 
%describing the curved edges describe approximately $\Theta = 360^{\circ}/(6\cdot 3) = 20^{\circ}$
%of the complete circle. 

An additional test performed on this case was how different settings in Nek5000 affect the estimation of the drag and lift coefficients.
Perhaps most curious is whether the $P_NP_N$ or $P_NP_{N-2}$ formulation is applied. Note that the pressure in the latter formulation is 
not defined on the boundary of the cylinder and does therefore need to be extrapolated onto the surface in order for the integral to be 
calculated. It is however more likely that this effect will be negligible compared with the splitting error implied by both schemes, 
since they both impose an erroneous boundary condition on the pressure.
%
\subsection{Results - benchmark comparison}
The results are presented in \tref{tab:testcase}, and they confirm that the treatment of the geometry is 
essential, both coefficients are computed with significantly better accuracy. 
%
\begin{table}[h]
\centering
\begin{tabular}{l l c c c c}
		\toprule
		\# of Cells & Software & $c_D$ & $c_L$ & \%\textbf{Err} $c_D$ &\%\textbf{Err} $c_L$ \\ \midrule 
		2124030& Nek5000 (mid) & 6.18349 & 0.008939 & 0.030 & 4.19 \\ 
		2124030& Nek5000 (arc) & 6.18498 & 0.009413 & 0.005 & 0.10 \\
		3145728 & CFX 		 & 6.18287 & 0.009387 & 0.04 &0.15 \\
		3145728 & OpenFOAM	 & 6.18931 & 0.00973 & 0.06 &3.5 \\
		3145728 & FEATFLOW   & 6.18465 & 0.009397 & 0.01 &0.05 \\
		\bottomrule	
	\end{tabular}
	\caption{Results for the drag and lift coefficients with reference values 
	$c_D = 6.18533$ and $c_L = 0.009401$. $P=11$ for the simulations in Nek5000.}
\label{tab:testcase}
\end{table}
%
Compared with the results from the other softwares applied in~\cite{benchmark} Nek5000 performs 
just as well or better in most cases. It should be mentioned that the division of the grid is created
in a different manner for Nek5000 so the comparison is not as direct as it may seem from the table.

\subsection{Results - internal adjustments }
As discussed in \cref{nek} there are many adjustments available in Nek5000. 
To understand the actual effect on the results, several different settings were 
investigated for this case and the results are presented in \tref{tab:perf}. 
The spectral convergence is also confirmed in \fref{fig:liftconv} by calculating the 
lift coefficient error for increasing polynomial degree. 
%
\begin{figure}[h]
	\centerline{
        \includegraphics[trim=0.5cm 7cm 0.5cm 7cm, width=0.8\textwidth]{Figures/lift_coef4.pdf}}
	\caption{The logarithm of the error plotted against the polynomial degree. All results 
        are with $P_NP_{N-2}$ and de-aliasing, and they are solved without using the 
    characteristic scheme or any filtering. Two lines illustrating a second order convergence and spectral convergence 
    is plotted to illustrate the convergence rate obtained in Nek5000.}
	\label{fig:liftconv}
\end{figure}
%

The setting that has the biggest impact on the result is the $P_NP_N$ scheme which clearly performs 
worse than the others. It would seem like the error on the boundary known to be present in both schemes
has a bigger effect in the $P_NP_N$ formulation.
Use of the IOFS method was expected to have a negative effect on the accuracy as stated in~\cite{Deville}, 
this is however not observed for this case.
The filtering does have a negative effect on calculation of the lift coefficient as expected,
but it is of relatively small significance which confirms the analytical results from \eref{eq:filterenergy}. 

Be aware that these results are obtained from a laminar test case and does not in any way 
suggest any optimal adjustment for Nek5000. It is however important to be aware that the activation 
of de-aliasing does not guarantee a better result. For a well-resolved flow situation such as $N=11$ 
for this case, the effect of aliasing is negligible and applying a higher order quadrature to resolve the 
non-linear term is a waste of computational time. For $N=7$ the results show that activation of de-aliasing
provides a better estimate, so the decision to activate de-aliasing or not depends on the type of flow
and the resolution of the mesh. The fact that the aliased solution for $N=11$ performs better than the rest, 
is probably because of the accuracy of the reference solution.
%
\begin{table}[h]
    \centering
    \begin{tabular}{c | c c c c c | c c }
         & \multicolumn{5}{|c|}{Settings} & \multicolumn{2}{|c}{\% Error} \\\hline
         \# & N & ifsplit & Dealiasing & IOFS & Filter & $c_D$ & $c_L$ \\  \hline 
         1 &11& No & Yes& No & No & 0.005 & 0.10\\
         2 &11& No & No & No & No & 0.005 & 0.03\\
         3 &11& No & Yes& No & Yes& 0.006 & 0.31\\
         4 &11& No & Yes& Yes& No & 0.005 & 0.11\\
         5 &11& Yes& Yes& No & No & 0.012 & 2.35\\
         6 &7 & No & Yes& No & No & 0.002 & 12.24\\
         7 &7 & No & No & No & No & 0.013 & 25.04\\

    \end{tabular}
    \caption{Test of solver settings in Nek5000.}
    \label{tab:perf}
\end{table}
%

When designing the mesh it is important to be aware of how the computational time depends on the polynomial 
degree. In user manual for Nek~\cite{Nek} it states that the computational time for one time step is 
of order $\mathcal{O}(EN^4) = \mathcal{O}(EP^4)$. This is confirmed in the left plot in~\fref{fig:comptime}.
Another important consequence of increasing the polynomial degree is that the time step decreases with a factor
$\mathcal{O}(P^2)$ due to the CFL-condition seen in~\eref{eq:restriction}. The computational time per 
time unit is therefore expected to be $\mathcal{O}(EP^{6})$ which is confirmed in the right plot in~\fref{fig:comptime}.

%
\begin{figure}[h]
\centering
  \centerline{
\begin{minipage}{.6\textwidth}
  \centering
  \includegraphics[trim=0.5cm 5cm 0.5cm 5cm,width=1.0\linewidth]{Figures/tperstep.pdf}
\end{minipage}
\begin{minipage}{.6\textwidth}
  \includegraphics[trim=0.5cm 5cm 0.5cm 5cm,width=1.0\linewidth]{Figures/tpers.pdf}
  %\captionof{figure}{A figure}
\end{minipage}%
}
  \caption{Computational time for one time step and for one second in Nek5000 as a function of the polynomial degree $P$. }
  \label{fig:comptime}
\end{figure}
%
%----------------------------------------------------------------------------------------
\chapter{Concluding remarks} \label{discussion}
Nek5000 has proven to give accurate results with a relatively coarse mesh as
presented in~\cref{results}. This is as expected since it is based on a higher order 
method known to yield great accuracy. As for the performance it is no doubt that the 
effort put into the efficiency of the code has paid off. The possibility to obtain 
results 5-10 times faster than similar softwares is an important factor to keep in mind 
when choosing which software to use. 
The polynomial degree is chosen by changing a single parameter, which makes performance tests
and accuracy adjustments simple to do. 

Since the mathematical formulation in Nek5000 is based on a tensor product of the basis functions 
in each direction it is limited to the use of hex-mesh. This is no problem for the geometries 
studied in this thesis, but for complex geometry the use of tetrahedral mesh is mandatory.

Another aspect worth noticing is the filtering procedure which has some similarities to variational multiscale
as presented in~\cref{physfilt}. It would be interesting to design some test cases to further 
investigate its properties.  

The fact that Nek5000 is an open-source code is also a huge advantage to other black-box solvers. 
Although the user community is not that large there are several committed users that provide 
their help on short notice. Many different examples are available, and the user guide 
contains a nice introduction with everything needed to get the program up and running. 
The documentation of variables and functions are however sometimes missing, which was 
the main motivation behind the creation of Appendix~\ref{AppendixB}. 


\section{Further work}
The surface projection could be expanded and improved in many ways. Making it iterative 
such that the points are projected more controlled towards the surface is one possibility. 
Experimentation with the weighting function and the distance norm should be done to see 
what provides best results. Treating the points along the edges differently, since they 
have a larger risk of being affected by the neighbouring element. Omitting the projection 
for certain points, especially those on edges that limits to other boundary conditions.
Restricting the relative or absolute change to avoid too much disturbance in the original mesh. 
These are just some ideas that might be worth having in mind if applying or expanding this method.

For the simple array case it was experimented with different meshes and Nek5000 seemed to 
have no problems with refined grids from ICEM. The refinement functionality does however not 
provide the same flexibility to adjust the number of elements. Perhaps even better results could 
have been achieved by making an o-grid around the most important part of the domain,
say for instance the box, $x/H < 3,\: |y|/H < 0.4,\: z/H < 0.15$ the nodes could have been 
distributed in a more economical fashion and an overall better result could have been achieved.

Finding out why the dynamic Smagorinsky model did not work for the $P_NP_{N-2}$ is also 
something that should be investigated further. As well as experimentation with 
other LES-models. 

The effect of the filter is also something that could be studied further, perhaps it would 
be possible to support the choice of alpha based on some analytical result.
