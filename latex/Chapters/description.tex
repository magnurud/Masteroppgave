% Chapter 2 - physics theory

\chapter{Problem description} % Main chapter title

\label{description} % Change X to a consecutive number; for referencing this chapter elsewhere, use \ref{ChapterX}

\lhead{Chapter 2. \emph{background on fluid dynamics}} % Change X to a consecutive number; this is for the header on each page - perhaps a shortened title

%----------------------------------------------------------------------------------------
%	SECTION 1
%----------------------------------------------------------------------------------------
This chapter will present the Incompressible Navier-Stokes equations and give a thorough 
analysis of the physical effects of each term. It will also present the idea behind turbulence 
modelling, and prepare the mathematical formulations which will be further analysed in \cref{numerics}.

\section{The Incompressible Navier-Stokes equations}

%The physics regarding fluids in motion are described mathematically by the Navier-Stokes (N-S) equations. 
%The equations can be derived in many ways, and it is referred to \cite{White} for a complete 
%description of the necessary assumptions and simplifications.
%The general idea is to conserve momentum and mass in a control domain providing a system of two equations.
%In this thesis only the incompressible N-S equations will be considered, 
With the assumption of an incompressible flow the conservation of mass 
results in a divergence free restriction on the velocity $\mathbf{u}$
and the incompressible N-S equations on a domain $\Omega\subset\mathbb{R}^d$ can be stated as  
%
\begin{align}
    \begin{split}
    \frac{\partial \mathbf{u}}{\partial t} + \mathbf{u}\cdot \nabla\mathbf{u} = 
    -\nabla p + \frac{1}{Re} \Delta\mathbf{u} + \mathbf{f}, \qquad 
    \nabla \cdot \mathbf{u} = 0, \qquad \text{ on } \Omega.
    \end{split}
    \label{eq:NS}
\end{align}
%
%These equations have been studied for centuries and different physical situations
%lead to different simplifications and sets of equations.
%Examples are the Euler equations,
%Stokes problem and Darcy flow.

A commonly studied simplification of this problem is the Stokes problem which
is achieved by omitting the convective term. This thesis will use the Stokes 
problem as a starting point for analysis of the full problem in \cref{stokes}.
Before attempting to solve these equations it is important to understand the role of each term 
and their mathematical influence on the problem. 
\begin{itemize}
    \item $\partial \mathbf{u} /\partial t$
     - The time-derivative of the flow, for a steady state flow this term will be equal zero.
             The discretization of this term is often based on some implicit scheme. 
    \item $\mathbf{u} \cdot \nabla \mathbf{u}$
     - The convective term, describes the transport due to the flow itself on each of its components. 
    The term is not present in Stokes problem.
    The mathematical operator corresponding to this term is non-linear and non-symmetric, and does therefore require an explicit scheme. 
    Even with linear advection the operator quickly leads to instability. 
%\item  $\nabla \cdot \tau$ 
       %- $\tau= -p + \nu[ \nabla \mathbf{u} + \nabla^T \mathbf{u}]$ is known as the Reynolds stress tensor for incompressible flows.
       %The term $\nabla \cdot \tau$ simplifies to $-\nabla p + \nu \Delta \mathbf{u}$ if the velocity is assumed to 
       %be divergence free and the viscosity is assumed to be constant. 
       %It is a symmetric tensor, hence another 6 unknowns is introduced.

    \item $\nabla p$
    - The pressure gradient, removal of this term results in a pure advection diffusion problem.

    \item $Re$ 
    - The Reynolds number defined as $UL/\nu$ where $U$ is the velocity scale, $L$ is the length scale and $\nu$ 
      is the viscosity of the fluid. $Re$ describes the relation between the largest length scales of the flow
      and the viscous length scales. Notice that for large Reynolds number the unstable non-linear term 
      dominates the transportation. Turbulent flows are characterized by high Reynolds number.

    \item $\Delta \mathbf{u}$ 
    - The diffusive term describes the natural diffusion of the fluid. The contribution of the diffusional term 
    is inversely proportional to the Reynolds number.
    The corresponding mathematical operator stabilizes the system and it is therefore generally easier
    to solve the N-S equations for high-viscosity fluids. It should be mentioned that this term is actually a simplification 
    of the Reynolds stress tensor which can be made under the assumption of incompressibility. The more general formulation which 
    will be used in \cref{LES} is $2 \nabla s_{ij}$, the tensor $s_{ij}$ is known as the strain-rate tensor. 

\item $\mathbf{f}$ 
    - General term describing external body forces such as gravity. For incompressible flow the
    gravity term is incorporated in the pressure term, $\nabla p = \nabla p + \rho \mathbf{g}$. 

    \item $\nabla \cdot \mathbf{u}$ 
    - The divergence free condition from the mass equation.
\end{itemize}
For large Reynolds number the huge span in length scales requires mesh that is both very fine and very large
if \eref{eq:NS} are to be solved exactly.
Because a fine mesh implies a high computational cost a direct numerical solution (DNS) is not feasible for 
problems of a certain geometrical complexity. There are many different approaches on how to resolve the turbulent term and in 
this thesis the main approach will be through Large Eddy Simulations (LES) which will be discussed 
in \cref{LES}. The Navier-Stokes equations can only be solved if the boundary conditions are given. 
The boundary conditions are not stated explicitly in \eref{eq:NS} because they depend on the physical 
situation and belong as a specification to each individual case. The next subsection gives a quick overview of the different 
boundary conditions applied in this thesis.
%
\subsection{Boundary conditions}
Depending on the kind of flow and geometry a particular problem different boundary conditions are applied. In this section 
$\mathbf{n}$ and $\mathbf{t}$ will denote the normal and tangent vector to the surface.
\colorbox{red}{tangent vector to a twodimensional surface??}
The boundary conditions applied for the cases investigated in this thesis will be given the names 
I,O,SYM and W for Inflow, Outflow, Symmetry and Wall. Their mathematical formulation and physical interpretation are given as 
\begin{itemize}
    \item I 
        - Inflow, defining the velocity field on the boundary. Mathematically this is equivalent to 
        non-homogeneous Dirichlet conditions. 
        \begin{align}
            \mathbf{u} = \mathbf{g}(\mathbf{x},t).
        \end{align}
    \item O 
        - Outflow, letting the fluid flow ``effortlessly'' out through the boundary. Formally stated as
        \begin{align}
            \frac{1}{Re} \nabla\mathbf{u}\cdot \mathbf{n}-p\mathbf{n}= 0.
        \end{align}
    \item SYM 
        - Symmetry, denying any flux through the boundary without disturbing the tangential velocity. Convenient
        to apply in an open channel where the stream wise direction is parallel to the boundary. Mathematically this is 
        described as 
        \begin{align}
            \mathbf{u}\cdot \mathbf{n} = 0, \qquad (\nabla\mathbf{u}\cdot \mathbf{t})\cdot \mathbf{n} = 0.
        \end{align}
    \item W 
        - Wall, Representing a physical object. Also known as the no-slip condition which is based on the assumption 
        that the fluid closest to the object moves with the same speed as the object itself. In this thesis all objects and 
        geometries is kept still so mathematically this is equivalent to 
        homogeneous Dirichlet conditions. 
        \begin{align}
            \mathbf{u} = 0.
        \end{align}

\end{itemize}
%
\subsection{Weak formulation of N-S}
%Let us first assume constant viscosity, enabling the simplification $\nabla \cdot \tau = -\nabla p + \nu \Delta \mathbf{u}$.
The numerical algorithms applied in this thesis requires a weak formulation of \eref{eq:NS}.
Before the weak form is derived a few operators will be defined in order to simplify the final 
expression.
%
\begin{align}
    ( \mathbf{u},\mathbf{v})_{L_2} &= \int_{\Omega}\mathbf{u} \cdot \mathbf{v} d\Omega\\
    \mathcal{A}(\mathbf{u},\mathbf{v}) &= (\nabla \mathbf{u},\nabla \mathbf{v})_{L_2}\\
    \mathcal{B}(\mathbf{u},p) &= -(\nabla \cdot \mathbf{u},p)_{L_2}\\
    \mathcal{C}(\mathbf{w};\mathbf{u},\mathbf{v}) &= (\mathbf{w}\cdot \nabla \mathbf{u},\mathbf{v})_{L_2}
    \label{eq:weakoperators}
\end{align}
%
A weak formulation is obtained by multiplying with a test function $\mathbf{v}$ and integrating over
the entire domain.
\begin{align}
    \begin{split}
        \int_{\Omega}\frac{\partial \mathbf{u}}{\partial t}\cdot\mathbf{v}d\Omega
        + \int_{\Omega}(\mathbf{u}\cdot \nabla)\mathbf{u}\cdot\mathbf{v}d\Omega
        &= -\int_{\Omega}\nabla p\cdot \mathbf{v} d\Omega 
        + \nu \int_{\Omega}\Delta\mathbf{u}\cdot\mathbf{v}d\Omega
        + \int_{\Omega}\mathbf{f},\mathbf{v}d\Omega \\
		\int_{\Omega}(\nabla \cdot \mathbf{u}) qd\Omega &= 0.
    \end{split}
	\label{eq:NSweak1}
\end{align}
Introducing the compact inner product notation and applying the divergence theorem on the right hand side of 
the first equation yields
\begin{align}
    \begin{split}
        (\frac{\partial \mathbf{u}}{\partial t},\mathbf{v})
        + (\mathbf{u}\cdot \nabla\mathbf{u},\mathbf{v})
        &= (\nabla \cdot \mathbf{v} , p ) 
        -\nu(\nabla \mathbf{u},\nabla \mathbf{v})
        + (\mathbf{f},\mathbf{v}) \\
		(\nabla \cdot \mathbf{u},q) &= 0.
    \end{split}
	\label{eq:NSweak}
\end{align}
%
The contributions from the boundary as a result from the application of the divergence theorem is included in the 
last force term.
%Writing the covective term in conservational form $\nabla \cdot \mathbf{u}\mathbf{u}$ and applying the 
%divergence theorem on the terms leaves us with 
%\begin{align}
    %\begin{split}
        %(\frac{\partial \mathbf{u}}{\partial t},\mathbf{v})
        %- (\mathbf{u}\mathbf{u},\nabla \cdot \mathbf{v})
        %&= +(p,\nabla \cdot \mathbf{v} ) 
        %+\frac{1}{Re}(\tau,\nabla \cdot \mathbf{v}), \\
		%-(\mathbf{u},\nabla p) &= 0.
    %\end{split}
	%\label{eq:NSweak}
%\end{align}
The choice of search spaces for the velocity and pressure will be justified in \cref{numerics}, and will 
just be stated here in order to present the weak formulation. Let $V \subset H^1(\Omega)^3$ and $Q \subset L^2(\Omega)$ 
be subspaces that include the boundary conditions given by any particular flow situation.
By using the notation introduced in \eref{eq:weakoperators} the weak formulation of the incompressible
N-S equations can be stated as: 

Find $(\mathbf{u}, p) \in V\times Q$ such that 
\begin{align}
    \begin{split}
        (\frac{\partial \mathbf{u}}{\partial t},\mathbf{v})
        + \mathcal{C}(\mathbf{u};\mathbf{u},\mathbf{v})
        &= -\mathcal{B}(\mathbf{v},p) 
        -\nu\mathcal{A}(\mathbf{u},\mathbf{v})
        + (\mathbf{f},\mathbf{v}), \\
        \mathcal{B}(\mathbf{u},q) &= 0,
    \end{split}
	\label{eq:NSweak}
\end{align}
$\forall\; (\mathbf{v}, q) \in V\times Q$.
%

In order to solve this equation numerically, everything has to be discretized and expressed 
using a particular set of basis functions. The idea is that the solution $(\mathbf{u},p)$ is approximated 
by a discretized solution $(\mathbf{u}_h,p_h)$ and the bilinear operators can be represented 
by matrices. The discretized system of equations can be stated as
%
\begin{align}
    M\frac{\partial \mathbf{u}_h}{\partial t} +C(\mathbf{u}_h)\mathbf{u}_h 
    &= D^Tp_h-\nu A\mathbf{u}_h +M\mathbf{f}_h,\\
    D\mathbf{u}_h &= 0.
    \label{eq:NSMatrixform}
\end{align}
%
%\section{Existence and uniqueness}
%The N-S equations are studied in detail through many years and a lot of mathematical 
%theory is developed trying to solve this. Because of its complexity most of the analysis 
%is performed on the parts of the equation and then later extrapolated onto the full 
%equation. In this section a introduction to this analysis performed on the Stokes 
%problem will be given. 

%\subsection{The Stokes problem}
%The Stokes problem is given as 
%\begin{align}
    %\begin{split}
        %\frac{\partial \mathbf{u}}{\partial t} - \nu \Delta \mathbf{u} + \nabla p &= 
    %\mathbf{f} , \\
		%\nabla \cdot \mathbf{u} &= 0.
    %\end{split}
	%\label{eq:Stokes}
%\end{align}
%For the sake of clarity only the steady Stokes problem will be considered, the stability regarding
%the evolution in time is a different analysis left for \cref{timeNS}.
%Transforming \eref{eq:Stokes} to the weak form and applying the notation given
%in the previous chapter leaves us with the formulation

%Find $(\mathbf{u}, p) \in H^1(\Omega)\times L^2(\Omega)$ such that 
%\begin{align}
    %\begin{split}
        %%(\frac{\partial \mathbf{u}}{\partial t},\mathbf{v})
        %\nu\mathcal{A}(\mathbf{u},\mathbf{v})
        %+\mathcal{B}(\mathbf{v},p) &= (\mathbf{f},\mathbf{v}) \\
        %\mathcal{B}(\mathbf{u},q) &= 0.
    %\end{split}
	%\label{eq:Sweak}
%\end{align}
%$\forall\; (\mathbf{u}, p) \in H^1(\Omega)\times L^2(\Omega)$.
%%

%\subsection{inf-sup condition}

\section{The passive scalar equation}
The N-S equations explains how a fluid will behave and solving it provides a complete pressure-velocity field of the 
domain of interest, but it does not provide the answer of how heat will distribute in this flow, or how a gas will spread.
The equation corresponding to this physical problem will be referred to as the passive scalar equation for any scalar 
$\phi$ and is stated as 
\begin{align}
    \begin{split}
        \rho c_p(\frac{\partial \phi}{\partial t} + \mathbf{u}\cdot \nabla\phi) 
        &= \nabla \cdot(k\nabla \phi)+ q_{vol}. 
    \end{split}
	\label{eq:PS}
\end{align}
The constants $k$ and $\rho c_p$ are interpreted depending on 
the scalar transported. For dispersion of neutral gas with $\phi$ as the volume concentration of the gas
they resemble the viscosity and mass flux. The last term on the right hand side $q_{vol}$ is the source term and is not 
included in this thesis since all gas enters the control volume as a boundary condition.

The passive scalar equation is solved as a Helmholtz problem, by applying a explicit scheme on the convection term and an implicit 
scheme on the diffusion term. This is similar to the discretization performed on the momentum equation which will be discussed in 
detail in \cref{numerics}. 
%\colorbox{green}{write more about this equation and how it is solved \ldots}

\section{Resolving the turbulent term using LES} \label{LES}
%\colorbox{green}{tydeliggjor filteret!, og bruk en annen $\tau$, filtervidde, tydeligg diff antakelse}
When DNS is not feasible due to high Reynolds number on large domains LES is one of the most powerful tools for simulating turbulent flows.
The idea is based on the fact that the small turbulent structures behave homogeneously and are therefore easy to model.
This way the larger structures driven by geometry, inflow conditions and external forces can be simulated using a coarser 
grid while the effect of the small structures are modelled. 
LES will be introduced here in a mathematical fashion, although in many practical cases the filter function is not 
well defined. The reason for this is that the grid itself is often considered a filter, with the 
grid size as the filter width. As pointed out by Carati et al.~\cite{Carati} the filter is in this case nothing else but 
numerical discretization error. 
%\colorbox{red}{for spectral element it can be considered as a low pass filter...??}

\subsection{Filter}
The idea behind LES starts with defining a filter, which separates the modelled structures from the resolved ones. 
A filter in its general mathematical form introduced by Leonard~\cite{Leonard} is given as 
\begin{align}
    U^r(\mathbf{x},t) = \int_{\Omega} G_r(\mathbf{r},\mathbf{x})U(\mathbf{x}-\mathbf{r},t)d\mathbf{r}
    \label{eq:filter}
\end{align}

%\colorbox{red}{Define G!! and filter with $l_r$ }

%For the test filter applied in the dynamic Smagorinsky case in the procedure described in \cref{filtering}
 
It is generally assumed that the filter commutes with 
the differential operators $\nabla$, $\Delta$ and $\partial / \partial t$. By applying the filter on the N-S equations
and making the given assumptions the filtered N-S equations can be stated as 
%
\begin{align}
    \begin{split}
        \frac{\partial \mathbf{u}^r}{\partial t} + \mathbf{u}^r\cdot \nabla\mathbf{u}^r
        &= -\nabla p^r +\nu\Delta \mathbf{u}^r+\mathbf{f}^r-\nabla \cdot \tau, \\
        \nabla \cdot \mathbf{u}^r &= 0.
    \end{split}
	\label{eq:NSfiltered}
\end{align}
%
Where $\tau$ in this case denotes the subgrid-scale (SGS) stress given as 
\begin{align}
    \tau_{ij}(u_i,uj) = (u_iu_j)^r -u_i^ru_j^r.
    \label{eq:sgstensor}
\end{align}
%
This tensor is a consequence of applying the filter on the non-linear advection term, 
and it is this tensor that is modelled by a subgrid-scale model. See~\cite{Pope} for a full derivation on the 
application of a filter on the momentum equation. 

\subsection{dynamic Smagorinsky-Lilly SGS-model}
The problem is now reduced to modelling the tensor $\tau_{ij}$, one of the most common SGS-models is the 
dynamic Smagorinsky-Lilly model which is the one applied in this thesis. The initial progress of this model was made by 
Lilly in 1967~\cite{Lilly67} suggesting the following model for the SGS-tensor
%
\begin{align}
    \begin{split}
    \tau_{ij} &= -2C_sl^2\mathcal{S}^rs_{ij}^r,\\
    s^r_{ij} &= \frac{1}{2}\left( \frac{\partial u^r_i}{\partial x_j} +
    \frac{\partial u^r_j}{\partial x_i}\right),\\
    \mathcal{S}^r &= \sqrt{2s^r_{ij}s^r_{ij}}.
    \end{split}
    \label{eq:boussinesq}
\end{align}
%
Where $l$ denotes the filter width, which for this thesis is equivalent to the grid size.
The resolved strain rate $s^r_{ij}$ can be calculated from the filtered velocity and the problem is now reduced to determining
the constant $C_s$. There were several attempts to determine this constant for the entire domain, but in lack of a general 
geometry independent constant Germano presented in 1991~\cite{Germano91} a dynamic subgrid-scale model was presented calculating a 
$C_d$ which replaces $C_sl^2$ in \eref{eq:boussinesq}. $C_d$ is called the dynamic Smagorinsky constant, and varies in both time and space.
The general idea is that $C_d$ is independent of the filter width and from this 
assumption an expression is developed for the dynamical constant.

Let $a,b$ denote two distinct filters with corresponding filter widths $l_a,l_b$. 
Throughout this thesis $l_a$ will be the grid size and $l_b\approx 2l_a$. Filter $b$ is
often referred to as the test filter and is only included to provide an estimation 
of the dynamic Smagorinsky constant. Remember that $l_a$ is the grid size while 
filter $b$ is obtained by applying the filter described in \cref{filtering} with $\alpha_i=1$ for the highest modes.
Let $\tau_{ij}$ and $T_{ij}$ denote the stresses based on single- and double filtering
operations on the N-S equations
\begin{align}
    \begin{split}
    \tau_{ij} = (u_iu_j)^a - u_i^au_j^a,\\
    T_{ij} = ((u_iu_j)^a)^b - (u_i^a)^b(u_j^a)^b.
    \end{split}
    \label{eq:stresstensors}
\end{align}
%By defining $a_{ij}=l_a^2\mathcal{S}^a = l_a^2\sqrt{2s^{a}_{ij}s^{a}_{ij}}$ the subgrid scale tensor can 
%be written as $\tau^a_{ij} = 2C_da_{ij} $. 
Applying the $b$ filter on the first tensor in \eref{eq:stresstensors} allows us to define 
a new tensor $L_{ij}$ which depends only on the $a$-filtered variables. The identity 
is known as the Germano identity and was first introduced in 1991~\cite{Germano91},
\begin{align}
    L_{ij} = T_{ij} - (\tau_{ij})^b
    = (u_i^au_j^a)^b - (u_i^a)^b(u_j^a)^b.
    \label{eq:germanoid}
\end{align}
This tensor now depends on the $a$-filtered solution and not the resolved 
one, hence the identity in \eref{eq:germanoid} provides a computable expression for $L_{ij}$.

Substituting the stress-tensors with their corresponding expression 
from \eref{eq:boussinesq} and assuming a dynamic constant unaffected by the filter 
one obtains an approximation for $L_{ij}$ which is also computable,
\begin{align}
    \begin{split}
L_{ij} &\approx 2C_s l_b^2 \mathcal{S}^{ab} s^{ab}_{ij}
        -2 (C_s l_a^2 \mathcal{S}^a s^a_{ij})^b\\
        &\approx 2C_sl_a^2[\lambda^2\mathcal{S}^{ab}s^{ab}_{ij} 
        - (\mathcal{S}^{a}s^{a}_{ij})^b]\\
        &= 2C_d M_{ij}.
        \label{eq:lillystress}
    \end{split} \\
    M_{ij} &= \lambda^{2}\mathcal{S}^{ab}s_{ij}^{ab} - (\mathcal{S}^as_{ij}^a)^b\\
        C_d &= C_sl_a^2\\
        \lambda &= l_b/l_a
    \label{eq:dynsmagderivation}
\end{align}
Minimizing the mean-square error between the exact $L_{ij}$ as expressed in \eref{eq:germanoid}
and the Boussinesq-based approximation 
in \eref{eq:lillystress} yields the best approximation 
for the dynamic Smagorinsky constant 
%
\begin{align}
    C_s &= \frac{M_{ij}L_{ij}}{2M_{kl}M_{kl}}.
    \label{eq:dynsmag}
\end{align}
%
Note that the double indices implies summing. This expression is however not a 
stable option and to deal with this most implementations apply some sort of mean or smoothening 
in either time and/or space when calculating the constant. In this thesis the smoothening is done in 
both time and space for the denominator and the numerator in \eref{eq:dynsmag}. Another interesting property 
is that the constant $C_d$ and the final SGS-tensor $\tau = -2C_d\mathcal{S}^as_{ij}^a$ are 
independent of the filter width, the only necessary variable is the coefficient $\lambda = l_b/l_a$.
The assumption made in this model is that turbulence behaves 
as diffusion, similar to the kinematic viscosity a turbulent viscosity $\nu_t$ is defined which for this case is given as 
$\nu_t = C_s\mathcal{S}^a$.

Let us end this section by stating the filtered N-S equations with the LES using 
dynamical Smagorinsky subgrid scale model, and remember that the diffusive term is written 
in general terms as $2\nabla \cdot \nu s_{ij}$.  
%
\begin{align}
    \begin{split}
        \frac{\partial \mathbf{u}^a}{\partial t} + \mathbf{u}^a\cdot \nabla\mathbf{u}^a
        &= -\nabla p^a+\mathbf{f}^a +2\nabla \cdot (\nu + \nu_t) s^a_{ij} \\
        \nabla \cdot \mathbf{u}^a &= 0.
    \end{split}
	\label{eq:NSLES}
\end{align}
%
Notice that if $v_t$ is a constant in the entire domain this equation would 
be equivalent to the one for a fluid with viscosity $\nu'= \nu + \nu_t$. The idea is 
that $\nu_t$ will be larger when the subgrid-structures are significant and closer to zero 
when the flow is laminar. This is just one of many types of models but is attractive due to its
simple derivation from physical principles.

%\subsection{Decoupling the pressure} 

%A common way to approach equation \eref{eq:NS} is to solve the pressure and velocity field
%seperatly. The mathematical reasoning for this approach is obtained by taking the 
%divergence on both sides, 
%\begin{align}
	%\nabla \cdot \frac{\partial \mathbf{u}}{\partial t} + 
    %\nabla \cdot(\mathbf{u} \cdot \nabla) \mathbf{u} 
    %&= -\Delta p + \nu \Delta \tau \\
	%\frac{\partial (\nabla \cdot  \mathbf{u})}{\partial t} + 
    %(\nabla \cdot \mathbf{u}) \cdot \nabla \mathbf{u} 
    %&= -\Delta p + \nu \Delta \tau \\
   %\Delta p &= \nu \Delta \tau \\
	%\label{eq:pressuredecoupling}
%\end{align}

%Where the divergence free criterium is applied to get to the final equation.
